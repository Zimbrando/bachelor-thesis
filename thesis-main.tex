\documentclass[12pt,a4paper,openright,twoside,table]{book}
\usepackage[utf8]{inputenc}
\usepackage{pgfplots}
\usepackage{disi-thesis}
\usepackage{code-lstlistings}
\usepackage{listings}
\usepackage{notes}
\usepackage{shortcuts}
\usepackage{acronym}
\usepackage{float}
\usepackage[normalem]{ulem}
\useunder{\uline}{\ul}{}

% Commands
\newcommand\tab[1][1cm]{\hspace*{#1}}
\newcommand{\imagesource}[5] {
	\begin{figure}[htb]
		\centering
		\includegraphics[width=#4\linewidth]{#1}
		\caption{#3}
		{\scriptsize%
			Source: \url{#2}}
		\label{fig:#5}
	\end{figure}
}

\school{\unibo}
\programme{Corso di Laurea in Ingegneria e Scienze Informatiche}
\title{Pacchettizzazione e Distribuzione Automatizzata di Software JVM-Based}
\author{Marco Sternini}
\date{\today}
\subject{Programmazione ad oggetti}
\supervisor{Dott. Danilo Pianini}
\cosupervisor{Dott.sa Martina Baiardi}
\session{III}
\academicyear{2022-2023}

% Definition of acronyms
\acrodef{vm}[VM]{Virtual Machine}
\acrodef{cicd}[CI/CD]{Continuous Integration Continuous Delivery}
\acrodef{cli}[CLI]{Command Line Interface}
\acrodef{jvm}[JVM]{Java Virtual Machine}
\acrodef{jre}[JRE]{Java Runtime Environment}
\acrodef{aur}[AUR]{Arch User Repository}
\acrodef{dsl}[DSL]{Domain-Specific Language}
\acrodef{kiss}[KISS]{Keep It Simple, Stupid}
\acrodef{dry}[DRY]{Don't Repeat Yourself}
\acrodef{jdk}[JDK]{Java Development Kit}
\acrodef{jit}[JIT]{Just in Time}
\acrodef{aot}[AOT]{Ahead of Time}

\mainlinespacing{1.241} % line spacing in mainmatter, comment to default (1)

\begin{document}

\frontmatter\frontispiece

\begin{abstract}	
Il software ha assunto un ruolo sempre più pervasivo nella vita quotidiana, portando con sé l'esigenza di sviluppare prodotti di qualità in modo rapido ed efficiente. Nel corso della storia, diverse metodologie di sviluppo si sono susseguite esponendo approcci diversi a un aspetto critico: il ciclo di vita di sviluppo del software (SDLC).  Recentemente, è emersa la filosofia ``DevOps", la quale ponendo al centro l'automazione, si prefigge di ridurre i tempi di sviluppo e migliorare la qualità del prodotto finale. 

L'obiettivo principale dell'elaborato è la realizzazione di un processo automatizzato per la distribuzione di un software JVM complesso, Alchemist, partendo dalla sua pacchettizzazione e concludendo con la sua pubblicazione all'interno di repository online per consentire il download agli utenti finali. La realizzazione del processo coinvolge due tecnologie ricorrenti nello sviluppo di software open-source, ossia Gradle come strumento di \textit{build automation} e GitHub Actions la piattaforma di \ac{cicd} utilizzata da Alchemist.
L'elaborato illustra l'intero procedimento: dall'analisi, fino al design dei componenti coinvolti e infine il percorso implementativo che ha permesso di sviluppare il processo richiesto.  Viene dedicata particolare attenzione alle sfide legate alla pacchettizzazione di software JVM; attraverso un'analisi approfondita degli strumenti disponibili nel panorama attuale, viene individuata la soluzione ottimale per assicurare la creazione di pacchetti di installazione funzionali e consistenti. Infine, si osserva il risultato ottenuto attraverso il confronto dei tempi di esecuzione di diverse versioni della pipeline, per valutare il lavoro svolto ed evidenziare le principali tecniche di ottimizzazione che le API di GitHub Actions consentono di utilizzare.

\end{abstract}

%----------------------------------------------------------------------------------------
\tableofcontents   
\listoffigures     % (optional) comment if empty
\lstlistoflistings % (optional) comment if empty
%----------------------------------------------------------------------------------------

\mainmatter

%!TEX root = ../thesis-main.tex

\chapter{Introduzione}\label{chap:introduction}
Il mondo del software ha scritto diverse decadi di storia. Sin dagli anni '50, quando i primi calcolatori programmabili hanno fatto il loro ingresso sul mercato, il software ha assunto un ruolo sempre più pervasivo nella vita quotidiana delle persone. Oltre ad essere parte integrante dei sistemi informativi delle aziende, lo possiamo trovare anche all'interno di automobili, elettrodomestici e tantissimi strumenti con la quale abbiamo a che fare nella nostra quotidianità. La crescente diffusione del software ha introdotto la necessità di progettare metodologie di sviluppo solide e versatili. Uno dei primi è il \textbf{modello a cascata} il quale struttura il processo di realizzazione del software in fasi sequenziali lineari. Il modello riprende la tipica organizzazione della produzione manifatturiera e fu progressivamente abbandonato con l'evolversi delle richieste del mercato. Successivamente prese piede il concetto di modelli iterativi come il \textbf{modello a spirale} in cui il processo di sviluppo è suddiviso in fasi multiple ripetute più volte (iterazioni). Gli ultimi decenni hanno dato vita a un nuovo modello, considerato lo standard dell'industria, la \textbf{metodologia agile}. Quest'ultima non rappresenta un unico modello, ma un insieme di modelli iterativi costruiti sulla base dei principi definiti all'interno del manifesto agile. Questi principi mettono in primo piano un ambiente autonomo e dinamico in cui sono fondamentali: cicli di sviluppo brevi, continui miglioramenti, la comunicazione col cliente e la consegna tempestiva di funzionalità. Il progetto esposto in questo documento introduce un evoluzione del concetto agile nato recentemente nel mondo dello sviluppo del software, conosciuto come ``DevOps".

\section{Contesto}
Con l'avvento di internet il concetto di software come un entità sviluppata e finita ha completamente cessato di esistere. Mediante la rete è diventato semplice ed efficiente distribuire un programma e fornire un ulteriore supporto attraverso aggiornamenti evolutivi e correttivi. Il fenomeno è cresciuto tanto da aver dato luce alla pratica del rilascio di applicazioni deliberatamente non complete, le quali attraverso il feedback degli utenti evolvono verso un prodotto finito. Il manifesto agile ha introdotto la cultura di emettere frequenti rilasci di nuove versioni del software, rendendo la distribuzione un punto cardine all'interno del ciclo di vita di esso. Dietro lo sviluppo rapido di nuove funzionalità è necessario il rilascio di queste altrettanto velocemente, la filosofia DevOps nasce per soddisfare questa esigenza.

\subsection{DevOps}
La metodologia DevOps (termine nato dalla contrazione di ``Development" ed ``Operations") si è formata intorno al 2008 con l'idea chiave di unire il team di sviluppo ed il team operativo. Il principale catalizzatore di questo concetto è stata la necessità di affrontare inefficienze nelle fasi del ciclo di vita del software. Differentemente dalla metodologia agile, DevOps è una filosofia di sviluppo software che esprime attraverso tre pilastri il suo obiettivo:

\begin{itemize}
	\item il \textbf{flusso}, il miglioramento del flusso di lavoro lungo l'intero processo di produzione, ciò significa ottimizzare il processo dall'idea, fino alla generazione di valore con il software in produzione.
	\item Il \textbf{feedback}, mediante cicli di feedback rapidi si garantisce la scoperta di difetti nel codice nelle fasi iniziali del ciclo di vita del prodotto. Ciò comporta rapide correzioni, minor debito tecnico e la garanzia di possedere in qualsiasi momento un software stabile e qualitativamente pronto ad un rilascio.
	\item L'\textbf{apprendimento continuo}, la filosofia DevOps promuove la sperimentazione continua, ossia interrogarsi regolarmente sui possibili miglioramenti attuabili assumendosi i rischi che l'applicazione di questi può recare.
\end{itemize}

Le nozioni fornite dalla cultura DevOps ricoprono diversi ambiti e non si limitano agli aspetti tecnici del ciclo di vita del software. Nella pratica esistono diverse tecnologie che concorrono allo sviluppo di processi conformi alla metodologia presentata.

\begin{figure}[htb]
	\centering
	\includegraphics[width=.8\linewidth]{figures/devops-process.png}
	\caption{Le fasi della metodologia DevOps}
	\label{fig:devops-process}
\end{figure}

Il modello illustrato nella figura \ref{fig:devops-process} rappresenta il ciclo di vita del software secondo la filosofia DevOps. La disposizione delle fasi, configurata in modo da evocare il simbolo dell'infinito, simboleggia il concetto di continuità fondamentale per questa metodologia. Questo concetto è introdotto all'interno del flusso mediante un altro pilastro: l'\textit{automazione}. Grazie all'automazione, gli sviluppatori possono delegare compiti complessi o ripetitivi a sistemi esterni configurando tre componenti chiave: un evento, un'azione e il risultato atteso. Quando si verifica un evento specifico, un'entità esterna esegue un insieme di azioni predeterminate, il cui successo o fallimento viene determinato dal confronto con il risultato atteso. A livello pratico, ciò è ottenuto mediante l'utilizzo di server o più generalmente infrastrutture cloud complesse.

L'automazione dunque garantisce l'esecuzione dei processi in modo consistente e permette di concentrare le risorse del team sullo sviluppo, eliminando quindi l'intervento umano da compiti ripetitivi e passibili di errori. Una delle pratiche più diffuse, concetto rilevante della filosofia DevOps, è la pipeline di \ac{cicd}.

\paragraph{Continuous integration} La pratica della \textit{continuous integration} si concentra sull'integrazione automatica e continua delle modifiche al codice sorgente del progetto. Tipicamente, il processo si articola nei seguenti passaggi: (i) gli sviluppatori introducono nuovo codice nel progetto attraverso il software di \textit{version control}, (ii) un server acquisisce le modifiche, compila e testa l'intero progetto, (iii) una volta completato il processo, comunica agli sviluppatori l'esito delle operazioni. Questo approccio consente di individuare errori nel codice anticipatamente, garantendo stabilità e una maggiore qualità al software.

Un aspetto fondamentale è la stesura dei test: un'eccessiva copertura può rallentare il processo di integrazione. È pertanto essenziale bilanciare la copertura dei test in base alle esigenze del progetto, tenendo presente che un aumento della copertura riduce il rischio di introdurre codice difettoso.

\paragraph{Continuous delivery} La distribuzione rappresenta l'insieme di operazioni finalizzate alla consegna del software agli utenti finali. Questo processo estende l'integrazione continua e si preoccupa di garantire la disponibilità costante di un artefatto di build pronto per il rilascio. L'effettivo rilascio di una nuova versione del software può avvenire in modo automatico oppure manualmente da parte dello sviluppatore. La filosofia DevOps fornisce linee guida e non regole rigide, lasciando al team di sviluppo il compito di progettare ed integrare un flusso adeguato alle necessità del progetto.

\subsection{Un software complesso: Alchemist}\label{sec:alchemist}
Alchemist\cite{Pianini_2013} è un framework di simulazione open-source sviluppato dall'Università di Bologna, progettato per modellare elementi di programmazione pervasiva. Per comprendere l'ambito del progetto è necessario introdurre il concetto di simulazione in ambito scientifico. Per simulazione si intende un modello della realtà, costruito secondo le regole di un analista, sviluppato per consentire la valutazione dello svolgersi di una serie di eventi in un ambiente definito. Lo svolgimento di una simulazione avviene all'interno di un arco di tempo discreto suddiviso in unità di tempo predefinite conosciute come \textit{step}. Alchemist, consente di creare, osservare ed analizzare simulazioni atte a modellare interazioni tra agenti autonomi in ambienti dinamici: ossia scenari di \textit{aggregate} e \textit{nature-inspired computing}. Una rappresentazione del meta-modello, ossia le entità e relazioni configurabili, è raffigurata nella \cref{fig:alchemist-metamodel}.

\imagesource{figures/alchemist-metamodel.pdf}{https://alchemistsimulator.github.io/explanation/metamodel/}{Il meta-modello di Alchemist}{.9}{alchemist-metamodel}

\paragraph{Architettura}
Il framework è realizzato mediante linguaggi JVM-based, più precisamente Java e Kotlin, utilizzando una struttura modulare ed estendibile. Il simulatore, come già citato, è un progetto open-source, ossia distribuito sotto termini di una licenza aperta. Questa permette a tutti di osservare il codice sorgente e di contribuire allo sviluppo del progetto, coordinato da un personale responsabile del suo avanzamento. La natura open-source del progetto apre le porte a modalità di sviluppo del codice differenti rispetto a team di dimensioni ridotte. In un progetto aperto, gli utenti che contribuiscono allo sviluppo sono potenzialmente infiniti, ragion per cui l'automazione risulta determinante per mantenere un processo ordinato e controllato di integrazione e rilascio del software.

\section{Tecnologie}\label{sec:technologies}

\subsection{Gradle}

Mentre in passato la produzione di artefatti (documentazione, pacchetti, eseguibili) era delegata a script costruiti dallo sviluppatore, in un progetto di grandi dimensioni è oggigiorno essenziale avvalersi di uno strumento di \textit{build automation}. Come l'output di un programma deterministico non cambia per uno stesso input, la produzione di artefatti deve essere consistente e riproducibile riducendo al minimo l'intervento umano. 

Gradle\footnote{https://gradle.org/} è uno dei tanti strumenti disponibili, supporta diversi linguaggi di programmazione anche se risulta popolare nell'ambiente JVM come alternativa a Maven. I \textit{task} sono l'unità minima di esecuzione e rappresentano un azione: come generare un JAR, eseguire dei test o produrre la documentazione. Mediante direttive come \textit{dependsOn} è possibile creare dipendenze tra processi: Gradle orchestra l'esecuzione dei task costruendo un grafo aciclico diretto (DAG) delle dipendenze. L'esecuzione di Gradle avviene in tre fasi distinte elencate di seguito.
\begin{enumerate}
	\item \textbf{Fase di inizializzazione}: in primo luogo Gradle crea un'istanza di Settings che organizza l'architettura del progetto. Attraverso un file, di nome ``settings.gradle", lo sviluppatore stabilisce il progetto radice e tutti gli eventuali progetti figli. 
	\item \textbf{Fase di configurazione}: successivamente tutti i file di configurazione ``build\-.\-gradle" (del progetto radice e tutti i sotto-progetti) vengono analizzati per costruire il grafo dei task.
	\item \textbf{Fase di esecuzione}: infine, Gradle esegue i task richiesti considerando le dipendenze descritte nel grafo generato nella fase precedente.
\end{enumerate}

\imagesource{figures/gradle-build-lifecycle-example.png}{https://docs.gradle.org/current/userguide/build_lifecycle.html}{Esempio di inizializzazione, configurazione ed esecuzione di una build Gradle}{1}{gradle-build-lifecycle}

Un componente chiave sono i plugin, i quali consentono di estendere le funzionalità di Gradle: aggiungere nuovi task, estendere il modello con nuovi elementi ed applicare configurazioni specifiche all'intero progetto. La presenza di diversi plugin base e creati dalla comunità rende Gradle uno strumento versatile.

\subsection{GitHub Actions}

Tramite Gradle lo sviluppatore è in grado di eseguire procedure articolate come compilazione, test e dispiegamento utilizzando un semplice comando da \ac{cli}. L'invocazione di queste procedure richiede però l'intervento umano, è quindi necessario uno strumento capace di automatizzare i processi offrendo un infrastruttura resiliente e facilmente accessibile.

GitHub Actions è una piattaforma di \ac{cicd} disponibile per i repository ospitati su GitHub, che consente la configurazione ed esecuzione di pipeline personalizzate, chiamate \textit{workflow}. Questi \textit{workflow} sono flussi di processi che consistono in un insieme di \textit{job} eseguiti sequenzialmente o parallelamente all'interno di macchine virtuali denominate \textit{runner}. I workflow (\cref{fig:github-actions-example}) sono descritti in file YAML all'interno di una cartella specifica del repository. Questi file definiscono i passaggi (\textit{step}) e le azioni (\textit{action}) che il runner deve eseguire all'interno di \textit{job}, macro-processi incaricati all'esecuzione sequenziale di step.

\imagesource{figures/overview-actions-simple.png}{https://docs.github.com/en/actions/learn-github-actions/understanding-github-actions}{Sintesi dei componenti utilizzati su GitHub Actions ed esempio di un workflow}{.9}{github-actions-example}

Uno step, come citato precedentemente, rappresenta l'unità minima di esecuzione all'interno della piattaforma Actions. Le API supportano due differenti tipologie di step: (i) le azioni, ossia componenti riutilizzabili e parametrizzabili delegati all'esecuzione di una procedura specifica e (ii) i comandi shell. Le azioni possono essere create personalmente o riutilizzate attingendo da un vasto marketplace manutenuto dalla comunità.  Ad esempio, una delle azioni più diffuse e utilizzate è ``actions/checkout" [\cite{github-actions-diffusion}], la quale clona il repository del progetto nella cartella di lavoro corrente del runner.

\subsection{Package manager}

Il \textit{package management system} è un insieme di strumenti software che gestiscono i processi di installazione, aggiornamento, configurazione e rimozione di applicativi dal sistema. Tipicamente, ogni pacchetto corrisponde ad un singolo programma o applicazione, tuttavia, esistono anche applicazioni più complesse composte da numerosi pacchetti correlati. Il sistema di gestione dei pacchetti opera attraverso tre componenti principali:

\begin{itemize}
	\item Un componente a basso livello, che si occupa principalmente dell'installazione o rimozione dei pacchetti. Ricoprono questo ruolo: \textbf{rpm} il quale gestisce gli omonimi pacchetti e dpkg relativamente alla tipologia \textbf{deb}.
	\item Un componente ad alto livello, il cui compito principale è quello di fornire un'interfaccia all'utente come: \textbf{yum} per Fedora o \textbf{apt} per Debian. Si occupa inoltre di risolvere le dipendenze e gestire le sorgenti esterne (repository).
	\item I repository, ossia archivi pubblici ospitati online dalla quale l'interfaccia ad alto livello ottiene i pacchetti ed i relativi meta-dati.
\end{itemize}

\begin{figure}[H]
	\centering
	\includegraphics[width=.7\linewidth]{figures/package-managers.pdf}
	\caption{Struttura più diffusa dei sistemi di gestione di pacchetti}
	\label{fig:package-managers}
\end{figure}

L'utilizzo dei package manager fornisce diversi vantaggi nella gestione di un sistema: semplifica notevolmente il processo di aggiornamento, in quanto mediante un comando è possibile aggiornare tutti i componenti, ed allo stesso tempo, funziona da strumento di installazione capace di reperire pacchetti distribuiti in archivi online ed installarli correttamente assieme alle dipendenze richieste. 

\section{Obiettivi}
I punti discussi precedentemente hanno evidenziato l'importanza che l'automazione ricopre all'interno dello sviluppo del software. Il rilascio continuo di un applicazione è necessario per mantenere elevati standard di qualità e comporta la necessità di automatizzare questi processi per garantire la loro esecuzione in modo consistente. Inoltre, la distribuzione dell'applicativo deve avvenire mediante mezzi compatibili e diffusi per assicurare un processo di installazione semplice e funzionale agli utenti. Il package management system in questo ambito offre un approccio valido: le diverse distribuzioni Linux adoperano esso dagli albori e nei restanti due sistemi operativi, Windows e MacOs, il suo utilizzo si sta diffondendo sempre maggiormente.
\begin{figure}[htb]
	\centering
	\includegraphics[width=.75\linewidth]{figures/use-case-diagram.pdf}
	\caption{Diagramma dei casi d'uso dallo sviluppatore all'utente}
	\label{fig:use-case-diagram}
\end{figure}

L'obiettivo principale dell'elaborato è quindi quello di progettare un sistema di pacchettizzazione e distribuzione del software automatico all'interno di una pipeline di integrazione e distribuzione continua. I nuovi processi devono integrarsi all'interno dell'attuale pipeline di Alchemist, estendendo le funzionalità di assemblaggio e rilascio del software. La distribuzione del pacchetto software deve avvenire per mezzo di repository pubblici diffusi, in modo da raggiungere il numero maggiore possibile di utenti; essa, inoltre, deve avvenire consistentemente nell'istante in cui una nuova versione di Alchemist viene rilasciata. La totalità dei processi integrati, inoltre, deve prevedere l'esecuzione di test all'interno del flusso, in modo da fornire un riscontro immediato nell'eventualità gli artefatti siano difettosi. Nella \Cref{fig:use-case-diagram} sono rappresentati gli scenari di utilizzo da parte degli utenti finali e degli sviluppatori di Alchemist. In particolare, lo sviluppatore introduce nuovo codice all'interno del repository del progetto e successivamente l'esecuzione della pipeline determina se è necessario il rilascio di una nuova versione. In caso di esito positivo, vengono generati i pacchetti software di installazione. Questi pacchetti, una volta distribuiti, consentiranno agli utenti di introdurre Alchemist nel proprio sistema.
%!TEXroot = ../thesis-main.tex

\chapter{Scenario di riferimento}

% Introduzione alle motivazioni che spingolo 'utilizzo di un software come Alchemist

\subsection{Un software complesso: Alchemist}\label{sec:alchemist}
Alchemist\cite{Pianini_2013} è un framework di simulazione open-source sviluppato dall'Università di Bologna, progettato per modellare elementi di programmazione pervasiva. Per comprendere l'ambito del progetto è necessario introdurre il concetto di simulazione in ambito scientifico. Per simulazione si intende un modello della realtà, costruito secondo le regole di un analista, sviluppato per consentire la valutazione dello svolgersi di una serie di eventi in un ambiente definito. Lo svolgimento di una simulazione avviene all'interno di un arco di tempo discreto suddiviso in unità di tempo predefinite conosciute come \textit{step}. Alchemist, consente di creare, osservare ed analizzare simulazioni atte a modellare interazioni tra agenti autonomi in ambienti dinamici: ossia scenari di \textit{aggregate} e \textit{nature-inspired computing}. Una rappresentazione del meta-modello, ossia le entità e relazioni configurabili, è raffigurata nella \cref{fig:alchemist-metamodel}.

\imagesource{figures/alchemist-metamodel.pdf}{https://alchemistsimulator.github.io/explanation/metamodel/}{Il meta-modello di Alchemist}{.9}{alchemist-metamodel}

\paragraph{Architettura} % Ulteriori spiegazioni magari sull'architettura già presente di una pipeline di continuouos integration
Il framework è realizzato mediante linguaggi JVM-based, più precisamente Java e Kotlin, utilizzando una struttura modulare ed estendibile. Il simulatore, come già citato, è un progetto open-source, ossia distribuito sotto termini di una licenza aperta. Questa permette a tutti di osservare il codice sorgente e di contribuire allo sviluppo del progetto, coordinato da un personale responsabile del suo avanzamento. La natura open-source del progetto apre le porte a modalità di sviluppo del codice differenti rispetto a team di dimensioni ridotte. In un progetto aperto, gli utenti che contribuiscono allo sviluppo sono potenzialmente infiniti, ragion per cui l'automazione risulta determinante per mantenere un processo ordinato e controllato di integrazione e rilascio del software.

\section{Tecnologie}\label{sec:technologies}

\subsection{Gradle}

Mentre in passato la produzione di artefatti (documentazione, pacchetti, eseguibili) era delegata a script costruiti dallo sviluppatore, in un progetto di grandi dimensioni è oggigiorno essenziale avvalersi di uno strumento di \textit{build automation}. Come l'output di un programma deterministico non cambia per uno stesso input, la produzione di artefatti deve essere consistente e riproducibile riducendo al minimo l'intervento umano. 

Gradle\footnote{https://gradle.org/} è uno dei tanti strumenti disponibili, supporta diversi linguaggi di programmazione anche se risulta popolare nell'ambiente JVM come alternativa a Maven. I \textit{task} sono l'unità minima di esecuzione e rappresentano un azione: come generare un JAR, eseguire dei test o produrre la documentazione. Mediante direttive come \textit{dependsOn} è possibile creare dipendenze tra processi: Gradle orchestra l'esecuzione dei task costruendo un grafo aciclico diretto (DAG) delle dipendenze. L'esecuzione di Gradle avviene in tre fasi distinte elencate di seguito.
\begin{enumerate}
	\item \textbf{Fase di inizializzazione}: in primo luogo Gradle crea un'istanza di Settings che organizza l'architettura del progetto. Attraverso un file, di nome ``settings.gradle", lo sviluppatore stabilisce il progetto radice e tutti gli eventuali progetti figli. 
	\item \textbf{Fase di configurazione}: successivamente tutti i file di configurazione ``build\-.\-gradle" (del progetto radice e tutti i sotto-progetti) vengono analizzati per costruire il grafo dei task.
	\item \textbf{Fase di esecuzione}: infine, Gradle esegue i task richiesti considerando le dipendenze descritte nel grafo generato nella fase precedente.
\end{enumerate}

\imagesource{figures/gradle-build-lifecycle-example.png}{https://docs.gradle.org/current/userguide/build_lifecycle.html}{Esempio di inizializzazione, configurazione ed esecuzione di una build Gradle}{1}{gradle-build-lifecycle}

Un componente chiave sono i plugin, i quali consentono di estendere le funzionalità di Gradle: aggiungere nuovi task, estendere il modello con nuovi elementi ed applicare configurazioni specifiche all'intero progetto. La presenza di diversi plugin base e creati dalla comunità rende Gradle uno strumento versatile.

\subsection{GitHub Actions}

Tramite Gradle lo sviluppatore è in grado di eseguire procedure articolate come compilazione, test e dispiegamento utilizzando un semplice comando da \ac{cli}. L'invocazione di queste procedure richiede però l'intervento umano, è quindi necessario uno strumento capace di automatizzare i processi offrendo un infrastruttura resiliente e facilmente accessibile.

GitHub Actions è una piattaforma di \ac{cicd} disponibile per i repository ospitati su GitHub, che consente la configurazione ed esecuzione di pipeline personalizzate, chiamate \textit{workflow}. Questi \textit{workflow} sono flussi di processi che consistono in un insieme di \textit{job} eseguiti sequenzialmente o parallelamente all'interno di macchine virtuali denominate \textit{runner}. I workflow (\cref{fig:github-actions-example}) sono descritti in file YAML all'interno di una cartella specifica del repository. Questi file definiscono i passaggi (\textit{step}) e le azioni (\textit{action}) che il runner deve eseguire all'interno di \textit{job}, macro-processi incaricati all'esecuzione sequenziale di step.

\imagesource{figures/overview-actions-simple.png}{https://docs.github.com/en/actions/learn-github-actions/understanding-github-actions}{Sintesi dei componenti utilizzati su GitHub Actions ed esempio di un workflow}{.9}{github-actions-example}

Uno step, come citato precedentemente, rappresenta l'unità minima di esecuzione all'interno della piattaforma Actions. Le API supportano due differenti tipologie di step: (i) le azioni, ossia componenti riutilizzabili e parametrizzabili delegati all'esecuzione di una procedura specifica e (ii) i comandi shell. Le azioni possono essere create personalmente o riutilizzate attingendo da un vasto marketplace manutenuto dalla comunità.  Ad esempio, una delle azioni più diffuse e utilizzate è ``actions/checkout" [\cite{github-actions-diffusion}], la quale clona il repository del progetto nella cartella di lavoro corrente del runner.

\subsection{Package manager}

Il \textit{package management system} è un insieme di strumenti software che gestiscono i processi di installazione, aggiornamento, configurazione e rimozione di applicativi dal sistema. Tipicamente, ogni pacchetto corrisponde ad un singolo programma o applicazione, tuttavia, esistono anche applicazioni più complesse composte da numerosi pacchetti correlati. Il sistema di gestione dei pacchetti opera attraverso tre componenti principali:

\begin{itemize}
	\item Un componente a basso livello, che si occupa principalmente dell'installazione o rimozione dei pacchetti. Ricoprono questo ruolo: \textbf{rpm} il quale gestisce gli omonimi pacchetti e dpkg relativamente alla tipologia \textbf{deb}.
	\item Un componente ad alto livello, il cui compito principale è quello di fornire un'interfaccia all'utente come: \textbf{yum} per Fedora o \textbf{apt} per Debian. Si occupa inoltre di risolvere le dipendenze e gestire le sorgenti esterne (repository).
	\item I repository, ossia archivi pubblici ospitati online dalla quale l'interfaccia ad alto livello ottiene i pacchetti ed i relativi meta-dati.
\end{itemize}

\begin{figure}[H]
	\centering
	\includegraphics[width=.7\linewidth]{figures/package-managers.pdf}
	\caption{Struttura più diffusa dei sistemi di gestione di pacchetti}
	\label{fig:package-managers}
\end{figure}

L'utilizzo dei package manager fornisce diversi vantaggi nella gestione di un sistema: semplifica notevolmente il processo di aggiornamento, in quanto mediante un comando è possibile aggiornare tutti i componenti, ed allo stesso tempo, funziona da strumento di installazione capace di reperire pacchetti distribuiti in archivi online ed installarli correttamente assieme alle dipendenze richieste. 

%!TEX root = ../thesis-main.tex

\chapter{Analisi}

\section{Requisiti}

Il progetto si pone due principali obiettivi:
\begin{itemize}
	\item La generazione automatica di pacchetti di installazione multi-piattaforma.
	\item La pubblicazione automatica dei rilasci del software all'interno di repository pubblici selezionati.
\end{itemize}
Un pacchetto software è un insieme di risorse necessarie per eseguire un'applicazione o un servizio su un sistema. Sono usualmente distribuiti all'interno di archivi compressi contenenti meta-dati che ne descrivono la forma e l'utilizzo. La pubblicazione è l'atto di inserire il software in archivi (repository) online con l'intento di consentire l'installazione agli utenti finali. Entrambi i processi devono essere automatizzati ed integrati all'interno di una pipeline di integrazione e rilascio continua.

\paragraph{Requisiti funzionali}

Le funzionalità richieste si possono classificare in due gruppi distinti: il primo contiene tutto ciò che concerne l'esperienza dell'utente finale, mentre il secondo descrive le funzionalità dal punto di vista degli sviluppatori e contributori di Alchemist. Di seguito il primo gruppo:
\begin{itemize}
	\item \textbf{Pacchettizzazione}: il simulatore deve essere distribuito in pacchetti \textit{self-contained}, ossia contenenti l'intera applicazione, e più specificatamente nell'ambito \ac{jvm}, integranti un \ac{jre} adibito all'esecuzione.
	\item \textbf{Multi-piattaforma}: Alchemist deve essere installabile sui maggiori sistemi operativi in circolazione come Windows, MacOS e le principali distribuzioni Linux.
	\item \textbf{Plug and play}: l'installazione non deve richiedere configurazioni complesse, l'applicativo deve essere pronto all'uso non appena installato.
\end{itemize}

I requisiti del gruppo successivo sono accumunabili per il loro scopo, vale a dire l'automazione:

\begin{itemize}
	\item \textbf{Automazione dei pacchetti}: la generazione dei pacchetti di installazione deve essere automatica e configurabile.
	\item \textbf{Automazione della distribuzione}: il rilascio di una nuova versione comprende la distribuzione di essa nei repository selezionati e deve essere svolta in modo automatico.
	\item \textbf{Verifica funzionamento}: entrambi i processi descritti precedentemente devono essere corredati da verifiche del loro funzionamento e devono bloccare la procedura di rilascio nell'eventualità siano presenti errori.
\end{itemize}

\paragraph{Requisiti non funzionali}

\begin{itemize}
	\item Trattandosi Alchemist di un software in continuo sviluppo, è auspicabile l'utilizzo degli strumenti già impiegati nel progetto. \\ L'integrazione di nuovi applicativi deve essere eseguita solo se strettamente necessaria.
	\item La pipeline \ac{cicd} ottenuta deve garantire prestazioni in linea con la versione precedente l'intervento di questo progetto.
\end{itemize}

\section{Pacchettizzazione}\label{sec:packaging}

Come si evince dall'analisi svolta, la pacchettizzazione è un requisito fondamentale dell'elaborato. Sino ad ora l'applicativo Alchemist era distribuito in file JAR, ossia archivi java compressi contenenti tutte le dipendenze ed i relativi \textit{classfiles} necessari all'esecuzione. Quest'approccio molto diffuso porta con sè diverse limitazioni tra cui:
\begin{itemize}
	\item la necessità dell'utente scaricante di avere un \ac{jre} installato nel proprio dispositivo;
	\item il potenziale bisogno di utilizzare un ambiente \ac{jre} specifico e quindi per l'utente di non possedere una versione dell'ambiente compatibile;
	\item la difficoltà di utilizzo per utenti non esperti, abituati ad eseguire programmi utilizzando file eseguibili nativi della propria piattaforma.
\end{itemize}
Alcune di queste restrizioni risultano mitigate, tuttavia non in tutte le piattaforme, come Windows, il quale per esempio non fornisce un ambiente java pre-installato. Esistono diversi strumenti di terze parti e non che cercano di far fronte a questo problema, è importante dunque valutare e scegliere lo strumento più opportuno.

\subsection{Soluzioni}

Nell'ottica di semplificare la configurazione di questo processo lo strumento selezionato deve supportare le tre principali piattaforme descritte nei requisiti precedentemente. Tra gli strumenti analizzati due molto differenti si sono distinti, vale a dire: \textit{jpackage}, comando disponibile nel \ac{jdk} dalla versione 14, adibito alla produzione di pacchetti self-contained con \ac{jre} integrata e \textit{GraalVM}, un \ac{jdk} sviluppato da Oracle che fornisce un compilatore \ac{jit} e \ac{aot} per java. 

La modalità \ac{jit} descrive il comportamento normale di ogni \ac{jvm}: il compilatore Java traduce il programma ad alto livello in bytecode, e successivamente la \ac{jvm} converte dinamicamente il bytecode in linguaggio macchina per l'architettura specifica. In contrasto la tecnica \ac{aot} ricorda i linguaggi di programmazione compilati, ovvero la compilazione è statica ed avviene prima dell'esecuzione del programma. Il compilatore \ac{aot} chiamato ``Native Image", consente di compilare un programma java (ed altri linguaggi di programmazione) ottenendo in output un eseguibile nativo per ogni piattaforma. Quest'ultimo inoltre porta con sè diversi vantaggi come: un minor costo in risorse CPU e memoria, tempi di avvio minori e dimensioni ridotte rispetto un normale programma java distribuito con un \ac{jre}. La compilazione \ac{aot} però ha dei requisiti, uno di questi fondamentale è la \textit{closed world assumption}: ossia ogni parte di codice raggiungibile in esecuzione lo deve essere anche in fase di build. Ciò accade perché native image svolge un analisi statica del codice, pertanto alcune funzionalità come la reflection oppure il caricamento dinamico non sono supportate e richiedono una soluzione alternativa.

In contrasto \textit{jpackage} fornisce uno strumento interessante, il quale non richiede modifiche all'applicativo ed è pre-installato in tutti i \ac{jdk} dalla versione 14 e successive. Lo strumento risolve le problematiche esposte inizialmente mediante l'annessione di un \ac{jre} utilizzando \textit{jlink}. Inoltre, produce diverse tipologie di pacchetti per ogni piattaforma ed è completamente controllabile da interfaccia \ac{cli}. Le tipologie di pacchetti generabili sono elencate di seguito:
\begin{itemize}
	\item \textit{exe} e \textit{msi} per Windows;
	\item \textit{rpm} e \textit{deb} per Linux;
	\item \textit{pkg} e \textit{dmg} per MacOs.
\end{itemize}
Tuttavia anch'esso presenta dei limiti come: la necessità di essere eseguito sullo stesso sistema operativo dove i pacchetti prodotti sono destinati (lo stesso vale per GraalVM, la \textit{cross compilation} non è supportata) e la produzione di un solo pacchetto per singola esecuzione.

\paragraph{Valutazione finale} Ambedue le soluzioni, seppure differenti concorrono all'obiettivo primario: la distribuzione del software multi-piattaforma. GraalVM fornisce diversi vantaggi prestazionali, ma i requisiti da esso richiesti non sono compatibili con l'architettura del simulatore. D'altro canto jpackage fornisce tutto il necessario per costruire i pacchetti con embed di un \ac{jre} senza la necessità di stravolgere l'architettura del software, ed i limiti delineati sono superabili adoperando script o configurazioni specifiche.

\section{Analisi degli strumenti}

Gli strumenti presentati nella sezione \ref{sec:technologies} costituiscono il progetto Alchemist nel suo complesso. Grazie agli script Gradle, il simulatore gestisce in modo efficace i processi di compilazione, generazione della documentazione ed esecuzione delle verifiche del codice. Mentre l'utilizzo della piattaforma Actions fornisce l'infrastruttura necessaria e le API per creare una pipeline di \ac{cicd}. 

\begin{figure}[htb]
	\centering
	\includegraphics[width=.6\linewidth]{figures/components-diagram.pdf}
	\caption{Diagramma dei componenti che interagiscono nel progetto}
	\label{fig:components-diagram}
\end{figure}

In considerazione dei requisiti posti, la pacchettizzazione risulta un processo di produzione di artefatti ed è perciò necessaria la sua integrazione nel build system. Mediante l'integrazione, si garantisce un corretto ordine di esecuzione rispetto ai diversi processi che il simulatore configura, con l'effetto di minimizzare l'incidenza di errori e assicurando la produzione di pacchetti conformi su qualsiasi sistema operativo in cui Gradle viene eseguito. Contemporaneamente Actions si occupa di offrire l'infrastruttura per automatizzare i processi esposti dal build system. La piattaforma fornisce le funzionalità necessarie a soddisfare i requisiti di automazione posti precedentemente, in particolare mediante: la possibilità di utilizzare macchine virtuali (runner) di tutte e tre i sistemi operativi target, la possibilità di configurare eventi o esecuzioni ricorrenti di processi e tramite la presenza di un'infrastruttura cloud resiliente per l'esecuzione della pipeline. Nel diagramma rappresentato nella \cref{fig:components-diagram}, sono illustrate le interazioni tra gli strumenti descritti. La pipeline, eseguita in un ambiente cloud, utilizza l'interfaccia fornita dal build system per generare il pacchetto di installazione e distribuirlo in modo che sia accessibile tramite i gestori di pacchetti designati.

\subsection{Distribuzione dei pacchetti}

La distribuzione ricopre un ruolo centrale nel processo di rilascio del software. L'obiettivo principale è estendere la disponibilità del simulatore Alchemist attraverso il maggior numero possibile di repository, al fine di permettere a più utenti possibili di installare e utilizzare il software. Nei prossimi paragrafi, verranno presentate due principali piattaforme di distribuzione coinvolte in questo progetto

\paragraph{Arch User Repository} Tra i numerosi fork Linux, Arch occupa una posizione di rilievo nel panorama. Basata sull'architettura x86-64, Arch Linux è stata sviluppata con l'adesione alla filosofia \ac{kiss}. Conosciuta per la sua leggerezza, velocità, e la sua estrema scalabilità, questa distribuzione si distingue per la sua capacità di adattarsi alle esigenze specifiche di ogni utente. Data la sua natura minimalista, l'installazione iniziale non incorpora alcun strumento di configurazione automatica, nessun ambiente desktop e nessun altro strumento necessario all'avvio del sistema. 

Il sistema di gestione dei pacchetti si chiama \textit{pacman} ed a differenza dei concorrenti, opera sia a basso che ad alto livello. Un pacchetto non è altro che un file shell script, denominato \textit{PKGBUILD}, contenente le istruzioni necessarie a scaricare i sorgenti e compilarli attraverso un comando: \textit{makepkg}. La linearità dei file PKGBUILD rende la creazione di pacchetti alla portata di qualsiasi utente, difatti Arch supporta l'\textbf{\ac{aur}}, un tratto distintivo di questa distribuzione. Si tratta di un repository di pacchetti in cui qualsiasi utente, anche non sviluppatore, può contribuire. Pertanto, la facilità di accesso e la mancanza di requisiti stringenti danno luce ad un ambiente perfetto per la distribuzione del simulatore.

\paragraph{Windows e winget} Un discorso differente vale per Windows, dove fino a poco tempo fa non era previsto alcun package manager ufficiale pre-installato. Gli utenti solitamente installavano software attraverso pacchetti distribuiti in siti web ad-hoc, store non ufficiali o tramite il Microsoft Store. D'altra parte gli sviluppatori, per sfruttare i benefici della gestione a pacchetti, ricorrono a gestori di terze parti. Solamente nel settembre 2020 è stato introdotto ``winget": un package-management system open-source sviluppato da Microsoft, che supporta pacchetti di installazione EXE, MSIX e MSI. Il repository dei pacchetti è accessibile pubblicamente ed è possibile mediante richieste di contribuzione e previa approvazione, pubblicare pacchetti all'interno di esso. \\

Poiché Windows e MacOs sono sistemi operativi closed-source, essi non richiedono particolari attenzioni in quanto le tipologie di pacchetti generabili da jpackage sono sufficienti ed ufficialmente supportate. Linux al contrario è notevolmente frammentato, ogni distribuzione può adottare uno dei tanti sistemi di gestione dei pacchetti o introdurne uno nuovo. Purtroppo, non esistono statistiche ufficiali riguardo la diffusione delle distribuzioni, molte di esse si basano su stime o dati non attendibili. D'altra parte, analizzando i pacchetti generabili da jpackage possiamo trarre delle conclusioni.
\begin{itemize}
	\item \textbf{RPM} significa "RedHat Package Manager" ed è un formato di pacchetti progettato per RedHat e le distribuzioni derivate. 
	\item \textbf{DEB} è l'abbreviazione di ``Debian packages", è una tipologia di pacchetto supportata da Debian e le distribuzioni derivate. Secondo distrowatch\footnote{https://distrowatch.com/}, Debian Linux presenta più di 400 distribuzioni derivate e più di 120 di queste sono attualmente attive.
\end{itemize}
È evidente come le due tipologie coprano un ampio spettro nel panorama Linux. Inoltre, esse sono supportate dallo script PKGBUILD nativamente, poiché il comando \textit{makepkg} supporta l'estrazione di questi pacchetti in modo autonomo. 

In conclusione, jpackage fornisce tutto il necessario per supportare in modo completo le piattaforme closed-source disponibili. Riguardo a Linux, i due tipi di pacchetti descritti svolgono un ruolo fondamentale nell'integrare diverse distribuzioni e rendere il software compatibile con l'\ac{aur}, contribuendo così ad ampliare la sua compatibilità nel vasto panorama delle distribuzioni Linux.

\input{chapters/4 - Design.tex}
\input{chapters/5 - Implementazione.tex}
%!TEX root = ../thesis-main.tex

\chapter{Conclusioni}
Quanto discusso nell'elaborato ha consentito la distribuzione automatica del software Alchemist nei formati standard delle piattaforme coinvolte, che rappresentano due famiglie di sistemi operativi completamente differenti. Gli obiettivi posti sono stati raggiunti grazie a una valutazione degli strumenti adibiti alla pacchettizzazione che il panorama \ac{jvm} offre, e grazie all'utilizzo di tecnologie come Gradle e GitHub Actions, sono state integrate all'interno del flusso di integrazione e distribuzione continua del progetto. L'autovalutazione ha poi evidenziato l'ottimo risultato ottenuto rispetto l'iniziale flusso di integrazione utilizzato dal software prima dell'intervento di questo progetto.

\begin{figure}[htb]
	\centering
	\includegraphics[width=.8\linewidth]{figures/alchemist-aur.png}
	\caption{Pagina web raffigurante il pacchetto Alchemist pubblicato sul repository AUR}
	\label{fig:aur-web}
\end{figure}

Il conseguimento degli obiettivi è osservabile su GitHub nella sezione dei rilasci\footnote{https://github.com/AlchemistSimulator/Alchemist/releases}, dove è possibile notare la possibilità di scaricare i diversi pacchetti installanti per ogni sistema operativo. Mediante invece l'utilizzo dei package manager è possibile installare ed aggiornare Alchemist con l'utilizzo di un semplice comando. Su Windows, previa installazione di winget, attraverso:
\texttt{winget install Unibo.alchemist}
e su Arch e derivate, previa autorizzazione all'installazione dall'Arch User Repository\footnote{https://aur.archlinux.org/packages/alchemist}(\Cref{fig:aur-web}), tramite \texttt{pamac} per Manjaro o più generalmente \texttt{yay}.
L'utilizzo dei package manager assicura all'utente l'installazione dell'ultima versione di Alchemist e mediante le funzionalità da questo offerte è altrettanto semplice aggiornare la sua versione, in modo da garantire l'utilizzo agli utenti delle ultime funzionalità del simulatore. 

\section{Sviluppi futuri}
L'implementazione descritta da questo elaborato ha aperto nuove possibilità di distribuzione del progetto mediante l'introduzione della pacchettizzazione. Attraverso lo sviluppo di processi automatici di pubblicazione il software è stato distribuito all'interno di due principali repository di riferimento. Le possibilità di estensione del progetto sono numerose, in quanto esistono molteplici package manager nel panorama esteso dei sistemi operativi. Nei seguenti punti sono riassunti i principali spunti per migliorare ed estendere il processo:
\begin{itemize}
	\item \textbf{Distribuzione su Homebrew}: i repository su cui Alchemist è distribuito non comprendono il sistema operativo MacOs. Il package manager Homebrew realizzato per MacOs rappresenta una valida opzione per raggiungere un pubblico più ampio attraverso modalità di installazione semplici e funzionali.
	\item \textbf{Supporto a Snap}: un'altra tipologia, nell'ambiente Linux, sono i pacchetti detti containerizzati, ovvero eseguiti all'interno di ambienti separati con un accesso limitato al sistema. Questa caratteristica fornisce due principali vantaggi: la possibilità per una applicazione di usare la propria versione desiderata di librerie di sistema senza creare conflitti e la trasparenza all'utente nell'accesso alle risorse di sistema, garantendo quindi un livello aggiuntivo di sicurezza. È il caso dei pacchetti \textit{snap}, pacchetti self-contained considerati universali perché compatibili con una notevole quantità di distribuzioni Linux. La loro implementazione nel flusso di integrazione e distribuzione continua di Alchemist contribuirebbe a un ulteriore ampliamento delle distribuzioni supportate dal software. 
\end{itemize}


%----------------------------------------------------------------------------------------
% BIBLIOGRAPHY
%----------------------------------------------------------------------------------------

\backmatter

\nocite{*} % comment this to only show the referenced entries from the .bib file

\bibliographystyle{alpha}
\bibliography{bibliography}

\end{document}