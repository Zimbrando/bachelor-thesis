%!TEXroot = ../thesis-main.tex

\chapter{Scenario di riferimento}\label{chap:scenery}

L'introduzione ha delineato gli obiettivi e le motivazioni che guidano il progetto, evidenziando il valore aggiunto che esso porta nell'ambito della distribuzione del software. Nel capitolo seguente, verrà presentato il software di riferimento oggetto dell'elaborato.

\section{Un software complesso}\label{sec:alchemist}
Il software selezionato per questo progetto è stato scelto in base a criteri fondamentali. In primo luogo, la sua complessità, in quanto garantisce che sia stato sviluppato in modo robusto e con elevati standard di qualità. In secondo luogo, la presenza di una solida infrastruttura di \ac{cicd}, poiché permette di focalizzare il progetto sulla pacchettizzazione e la presenza di un flusso ben testato pone le basi per integrare i processi di distribuzione richiesti. Il terzo requisito richiede che l'applicativo sia stato sviluppato specificamente per essere eseguito su una \ac{jvm}, al fine di esplorare le diverse metodologie di pacchettizzazione in relazione alle problematiche introdotte.

\subsection{Alchemist}

Un software che soddisfa ampiamente le richieste è Alchemist\cite{Pianini_2013}, un framework di simulazione stocastico open-source sviluppato dall'Università di Bologna, progettato per modellare elementi di programmazione pervasiva. Quest'ultima si riferisce a un paradigma in cui la computazione è integrata ovunque in modo onnipresente e invisibile (\textit{ubiquitous computing}). 

\imagesource{figures/alchemist-metamodel.pdf}{https://alchemistsimulator.github.io/explanation/metamodel/}{Il meta-modello di Alchemist}{.9}{alchemist-metamodel}

Il software si presenta come un simulatore, perciò per comprendere il suo utilizzo è necessario introdurre il concetto di simulazione in ambito scientifico. Per simulazione si intende un modello della realtà, costruito secondo le regole di un analista, sviluppato per consentire la valutazione dello svolgersi di una serie di eventi in un ambiente definito. Mediante la definizione delle entità e le regole che coinvolgono il modello studiato, l'analista attraverso Alchemist crea e osserva simulazioni atte a modellare interazioni tra agenti autonomi in ambienti dinamici: ossia scenari di \textit{aggregate} e \textit{nature-inspired computing}. Il meta-modello di Alchemist descrive gli elementi e le relazioni con cui l'utente finale è capace di costruire simulazioni di scenari complessi; come si può notare dalla \cref{fig:alchemist-metamodel}, le entità in gioco sono ispirate ai fondamenti della chimica. 

\paragraph{Architettura} Un aspetto importante per lo svolgimento del progetto è l'architettura del software in oggetto, la quale deve soddisfare i requisiti posti inizialmente. In questo contesto il framework propone una struttura consona, in quanto è realizzato mediante linguaggi JVM-based, più precisamente Java e Kotlin, e utilizza un'architettura modulare ed estendibile, sinonimo di robustezza. 

Il simulatore, come già citato, è un progetto open-source, ossia distribuito sotto termini di una licenza aperta. Questa permette a chiunque di osservare il codice sorgente e di contribuire allo sviluppo del progetto, coordinato da un personale responsabile del suo avanzamento. La natura open-source del progetto apre le porte a modalità di sviluppo del codice differenti rispetto a team predefiniti di dimensioni ridotte; in un progetto aperto, gli utenti che contribuiscono allo sviluppo sono potenzialmente infiniti, ragion per cui l'automazione risulta determinante per mantenere un processo ordinato e controllato di integrazione e rilascio del software. Alchemist, in quest'ottica, presenta un'infrastruttura di \ac{cicd} solida, nello specifico integra:
\begin{enumerate}
	\item il test e l'analisi statica del codice;
	\item la costruzione automatica di artefatti come la documentazione;
	\item il versioning automatico del software a ogni rilascio;
	\item la pubblicazione, mediante le release GitHub, degli archivi JAR dell'applicazione.
\end{enumerate}
L'insieme di queste azioni (a eccezione del rilascio, il quale avviene sotto certe condizioni) è eseguito in modo continuo ogni qualvolta del codice è introdotto nel repository: tramite una pull request esterna o attraverso una modifica diretta svolta dal maintainer del progetto.

\paragraph{Modalità di utilizzo} Il simulatore offre due modalità di utilizzo concepite per soddisfare esigenze diverse. 
Come chiaramente illustrato nella documentazione\footnote{https://alchemistsimulator.github.io/howtos/preparation/jar/index.html}, la modalità di utilizzo consigliata coinvolge l'impiego di Gradle. Quest'ultimo consente all'utente, generalmente un programmatore o un utente con competenze avanzate, di sfruttare Alchemist al massimo delle sue potenzialità; tale approccio risulta quindi più indicato quando è necessario avere un controllo completo per sviluppare simulazioni complesse. Il secondo approccio coinvolge l'utilizzo della \ac{cli} per avviare e osservare una simulazione di Alchemist. Mediante l'interfaccia a linea di comando si fornisce al simulatore un file descrittivo che elenca le entità e le regole della simulazione utilizzando gli elementi descritti nel meta-modello. Successivamente l'applicazione avvia l'esecuzione e l'utente può, secondo modalità scelte da lui stesso, osservare graficamente l'evolversi della simulazione.

L'obiettivo dell'elaborato è migliorare il processo di installazione e di onboarding del software, focalizzandosi in particolare sulla seconda modalità di utilizzo. Per fare ciò, si prevede di distribuire pacchetti di installazione \textit{self-contained}, eliminando così la dipendenza da risorse esterne che gli archivi JAR comportano. Inoltre, si intende fornire supporto ai package manager, al fine di semplificare ulteriormente il processo di download e aggiornamento del software.

\section{Tecnologie}\label{sec:technologies}

Alchemist utilizza diversi strumenti per sviluppare la sua infrastruttura di \ac{cicd}. Nella seguente sezione sono esposti i principali, i quali sono protagonisti del progetto esposto in questo elaborato.

\subsection{Gradle}

Mentre in passato la produzione di artefatti (documentazione, pacchetti, eseguibili) era delegata a script costruiti dallo sviluppatore, in un progetto di grandi dimensioni è oggigiorno essenziale avvalersi di uno strumento di \textit{build automation}. Come l'output di un programma deterministico non cambia per uno stesso input, la produzione di artefatti deve essere consistente e riproducibile riducendo al minimo l'intervento umano. 

Gradle\footnote{https://gradle.org/} è uno dei tanti strumenti disponibili, supporta diversi linguaggi di programmazione, anche se risulta popolare nell'ambiente JVM come alternativa a Maven. I \textit{task} sono l'unità minima di esecuzione e rappresentano un azione: come generare un JAR, eseguire dei test o produrre la documentazione. Mediante direttive come \textit{dependsOn} è possibile creare dipendenze tra processi: Gradle orchestra l'esecuzione dei task costruendo un grafo aciclico diretto (DAG) delle dipendenze. L'esecuzione di Gradle (\cref{fig:gradle-build-lifecycle}) avviene in tre fasi distinte elencate di seguito:

\imagesource{figures/gradle-build-lifecycle-example.png}{https://docs.gradle.org/current/userguide/build_lifecycle.html}{Esempio di inizializzazione, configurazione ed esecuzione di una build Gradle}{1}{gradle-build-lifecycle}

\begin{enumerate}
	\item \textbf{Fase di inizializzazione}: in primo luogo Gradle crea un'istanza di Settings che organizza l'architettura del progetto. Attraverso un file, di nome ``settings.gradle", lo sviluppatore stabilisce il progetto radice e tutti gli eventuali progetti figli. 
	\item \textbf{Fase di configurazione}: successivamente tutti i file di configurazione ``build\-.\-gradle" (del progetto radice e tutti i sotto-progetti) vengono analizzati per costruire il grafo dei task.
	\item \textbf{Fase di esecuzione}: infine, Gradle esegue i task richiesti considerando le dipendenze descritte nel grafo generato nella fase precedente.
\end{enumerate}

Un componente chiave sono i plugin, i quali consentono di estendere le funzionalità di Gradle: aggiungere nuovi task, estendere il modello con nuovi elementi e applicare configurazioni specifiche all'intero progetto. La presenza di diversi plugin base e altri creati dalla comunità rende Gradle uno strumento molto versatile. Nel progetto Alchemist, Gradle è utilizzato in modo esteso e costituisce il build system del software, si occupa inoltre di gestire le dipendenze, costruire gli artefatti ed eseguire gli unit test.

\subsection{GitHub Actions}

Tramite Gradle lo sviluppatore è in grado di eseguire procedure articolate come compilazione, test e dispiegamento utilizzando un semplice comando da \ac{cli}. L'invocazione di queste procedure richiede però l'intervento umano, è quindi necessario uno strumento capace di automatizzare i processi offrendo un infrastruttura resiliente e facilmente accessibile.

GitHub Actions è una piattaforma di \ac{cicd} disponibile per i repository ospitati su GitHub, che consente la configurazione ed esecuzione di pipeline personalizzate, chiamate \textit{workflow}. Questi \textit{workflow} sono flussi di processi che consistono in un insieme di \textit{job} eseguiti sequenzialmente o parallelamente all'interno di macchine virtuali denominate \textit{runner}. I workflow (\cref{fig:github-actions-example}) sono descritti in file YAML all'interno di una cartella specifica del repository. Questi file definiscono i passaggi (\textit{step}) e le azioni (\textit{action}) che il runner deve eseguire all'interno di \textit{job}, macro-processi incaricati all'esecuzione sequenziale di step.

\imagesource{figures/overview-actions-simple.png}{https://docs.github.com/en/actions/learn-github-actions/understanding-github-actions}{Sintesi dei componenti utilizzati su GitHub Actions ed esempio di un workflow}{.8}{github-actions-example}

Uno step rappresenta l'unità minima di esecuzione all'interno della piattaforma Actions. Le API supportano due differenti tipologie di step: (i) le azioni, ossia componenti riutilizzabili e parametrizzabili delegati all'esecuzione di una procedura specifica e (ii) i comandi shell o script. Le azioni possono essere create personalmente o riutilizzate attingendo da un vasto marketplace manutenuto dalla comunità. Per esempio, una delle azioni più diffuse è ``actions/checkout" [\cite{github-actions-diffusion}], la quale clona il repository del progetto nella cartella di lavoro corrente del runner.
