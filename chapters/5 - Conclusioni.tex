%!TEX root = ../thesis-main.tex

\chapter{Conclusioni}
% Dimostrare i link con le release, installazione da AUR e da Winget


\section{Sviluppi futuri}

L'implementazione descritta da questo elaborato ha consentito la distribuzione del software in diversi 

\begin{itemize}
	\item \textbf{Distribuzione su Homebrew}: le piattaforme su cui Alchemist è distribuito non comprendono il sistema operativo MacOs. Ciò è dovuto alla necessità di utilizzare un dispositivo che utilizza il sistema operativo per verificare la corretta integrazione. Il package manager Homebrew rappresenta una valida opzione per ampliare maggiormente il bacino di utenti di Alchemist.
	\item \textbf{Reflection e jlink}: l'utilizzo di Jlink
	\item \textbf{Supporto a Snap e Flatpak}: i pacchetti containerizzati offrono una valida alternativa e coprono una vasta gamma di distribuzioni dell'ambiente Linux. La loro implementazione nel flusso di integrazione e distribuzione continua di Alchemist contribuirebbe ad un ampliamento delle piattaforme supportate dal software. % Add stuff
\end{itemize}