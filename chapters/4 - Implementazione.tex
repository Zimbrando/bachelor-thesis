%!TEX root = ../thesis-main.tex

% Percorso implementativo:
% 	- jpackage e le opzioni in dettaglio
%  	- jlink e jdeps
%   - il plugin gradle per operare con jpackage
%   - la necessità di estendere con jpackageFull
%   - inserimento dei test e della release dei pacchetti
%   - le operazioni per la distribuzione e il test di queste (AUR)

\chapter{Implementazione}

Nel seguente capitolo è illustrato il percorso e le relative scelte implementative effettuate per soddisfare i requisiti posti dal progetto. Successivamente, valuterò il lavoro svolto considerando i requisiti non funzionali posti.

\section{Impacchettamento}

Come discusso nelle precedenti sezioni riguardanti lo strumento jpackage (in particolare sezione \ref{sec:design-jpackage}), esso presenta una notevole quantità di opzioni sia inerenti l'aspetto estetico, sia riguardo aspetti tecnici che possono contraddistinguere la qualità finale del pacchetto di installazione.
\begin{table}[]
	\centering
	\begin{tabular}{lcc}
		\rowcolor[HTML]{ECF4FF} 
		{\ul Opzione} & {\ul Funzionalità} & {\ul Eventuali parametri} \\
	    - -type & Permette di indicare le funzionalità richieste &  \\
		- -temp &  &  \\
		- -input	& & \\	      
		- -main-jar & & \\
		- -main-class & & \\
		- -add-modules & & \\                          
		- -jlink-options & & \\
		- -runtime-image & & \\
	\end{tabular}
	\caption{Tabella che elenca le principali opzioni tecniche del comando \texttt{jpackage}}
\end{table}

\paragraph{Ottimizzazione} Per comprendere le scelte implementative riguardo opzioni tecniche è necessario introdurre altri due strumenti del \ac{jdk}: (i) \texttt{jlink}, strumento delegato alla produzione della \textit{runtime-image} inserita all'interno del pacchetto e (ii) \texttt{jdeps}, comando utilizzabile per analizzare le dipendenze di un archivio JAR. I due strumenti assieme forniscono la capacità di ottimizzare lo spazio, generando runtime-image di dimensioni ridotte comprendendo all'interno della JVM solamente i moduli necessari all'esecuzione dell'applicazione. Per sfruttare questo potenziale è richiesto l'utilizzo dei tre strumenti a catena come esposto di seguito:
\begin{enumerate}
	\item jdeps, analizza l'archivio contenente la nostra applicazione e restituisce i moduli jvm necessari per il corretto funzionamento dell'applicazione. Il comando viene lanciato con i seguenti parametri
	\item jlink, utilizza i moduli evidenziati da jdeps per creare una runtime-image personalizzata di dimensioni ridotte
	\item jpackage, utilizza la runtime-image creata precedentemente assieme all'archivio dell'applicazione per generare il pacchetto: l'application-image.
\end{enumerate}

\subsection{JPackage}

\subsection{Automazione}

\section{Distribuzione}

\section{Valutazione}

- Cosa cambiare, ottimizzazioni