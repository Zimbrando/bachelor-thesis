%!TEX root = ../thesis-main.tex
\chapter{Design}

\section{Architettura e macrostruttura}

\subsection{Pipeline}

Alchemist, come esplicato nella sezione 1.4, è un progetto modulare complesso in continuo sviluppo. Gli strumenti presentati nel capitolo precedente trovano già impiego nel progetto, ad eccezione del software di impacchettamento il quale è oggetto di questo elaborato. Di seguito è illustrata l'architettura della pipeline principale e l'interazione tra Gradle e GitHub Actions.
\paragraph{Architettura iniziale} Alchemist utilizza fondamentalmente due workflow per soddisfare i requisiti di integrazione e rilascio continuo. Un primo workflow chiamato ``dispatcher", il quale funziona come punto di inizio del flusso, è in ascolto di diversi eventi e una volta avviato esamina (inspect) i meta-dati dell'evento scatenante per determinare lo step successivo. In particolare nella configurazione in oggetto questo step iniziale funge da filtro dal momento che esiste un solo possibile step successivo, ossia il secondo workflow quello adibito a soddisfare i compiti di build, test e distribuzione.


Il secondo e principale workflow costituisce il processo \ac{cicd} del progetto ed è formato dai seguenti job.
\begin{itemize}
	\item \textbf{select-java-version}: ottiene la versione di java di riferimento dalle proprietà Gradle del progetto e lo fornisce in output agli altri job.
	\item \textbf{build}: esegue un'analisi statica del codice ed effettua una build del progetto. 
	\item \textbf{test-deploy}: esegue un test sul deploy dei moduli di Alchemist.
	\item \textbf{build-website}: realizza il sito web di Alchemist.
	\item \textbf{release}: esegue il rilascio se è necessario.
	\item \textbf{success}: controlla che l'output di ogni job sia esente da errori.
\end{itemize}

\paragraph{}

\subsection{Task e interazione con Gradle}

\section{Flusso di rilascio}
Uno dei requisiti del metodo di sviluppo \ac{cicd} è 
