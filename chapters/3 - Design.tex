%!TEX root = ../thesis-main.tex
\chapter{Design}
Alchemist, come esplicato nella sezione \ref{sec:alchemist}, è un software modulare complesso in continuo sviluppo, gli strumenti presentati nel capitolo precedente trovano già impiego nel progetto, ad eccezione del software di impacchettamento il quale è oggetto di questo elaborato. Di seguito sarà illustrata l'architettura e l'interazione tra i componenti principali che compaiono nel processo di automazione.

\section{Architettura e macrostruttura}
L'analisi del progetto espone il coinvolgimento di tre diversi componenti per conseguire gli obiettivi di automazione e distribuzione del software. I tre componenti sono definiti come segue: 
\begin{itemize}
	\item \textbf{Build system}: l'insieme dei processi e delle funzioni adibite alla produzione di artefatti. In particolare questo componente richiede lo sviluppo di nuovi processi destinati alla produzione dei pacchetti, test di quest'ultimi e costruzione dei metadati necessari alla distribuzione del software.
	\item \textbf{Pipeline}: il processo automatizzato adibito alla gestione del flusso di lavoro del software, dalla compilazione al rilascio. Questo componente è già impiegato all'interno del progetto Alchemist per gestire l'attuale processo di integrazione continua, il design discusso in questo capitolo inserisce nuovi step all'interno del processo.
	\item \textbf{Release}: l'insieme delle procedure necessarie per la pubblicazione del software nel rispetto delle regole vigenti negli specifici repository pubblici.
\end{itemize}
I componenti interagiscono come raffigurato nella figura \ref{fig:activity-interaction-diagram} La pipeline sfrutta il build system per generare i pacchetti di installazione del software e integra tutti i processi necessari alla distribuzione negli specifici repository pubblici, producendo quindi l'automazione desiderata.
\begin{figure}[htb]
	\centering
	\includegraphics[width=.9\linewidth]{figures/activity-interaction-diagram.pdf}
	\caption{Diagramma delle attività raffigurante l'interazione tra i componenti}
	\label{fig:activity-interaction-diagram}
\end{figure}

Nello schema si fa riferimento ad un generico evento come segnale di avvio della pipeline. Nell'ambito \ac{cicd} l'evento corrisponde spesso alla pubblicazione di nuovo codice nel repository, in questo modo l'eventuale inserimento di errori viene rilevato e comunicato prontamente ai responsabili dello sviluppo del progetto.

\section{Build system}

Il ruolo principale del \textit{build system} è quello di esporre un \textit{task} adibito alla generazione dei pacchetti. Il task dovrà soddisfare i seguenti requisiti:
\begin{itemize}
	\item Deve configurare correttamente le opzioni di \textit{jpackage}, in particolare quelle che mutano nel tempo come la versione.
	\item È necessario stabilire un corretto ordine di esecuzione ed albero delle dipendenze per garantire la consistenza del processo.
	\item Deve consentire la dichiarazione di configurazioni differenti a seconda del sistema operativo che esegue il processo.
\end{itemize}

\subsection{Lo strumento jpackage}\label{sec:design-jpackage}
Lo strumento \ac{cli} \textit{jpackage} offre un'interfaccia munita di diverse opzioni per configurare e personalizzare a piacimento i pacchetti in output. Esistono parametri generici, compatibili con tutte le piattaforme, e parametri specifici che vanno a modificare attributi particolari alla tipologia di pacchetto in output scelta. Uno dei motivi che ha portato alla scelta di \textit{jpackage} rispetto ad altri software è la capacità di includere autonomamente una \textit{runtime-image} di Java, ossia una \ac{jre} ridotta di dimensioni all'interno del pacchetto. La combinazione di una \textit{runtime-image} e degli archivi Java (JAR) necessari all'esecuzione dell'applicazione costituiscono l'\textit{application-image}: un pacchetto autocontenuto che include l'applicazione, assieme una \ac{jvm} ed alle librerie necessarie per eseguire quell'applicazione sulla piattaforma di destinazione.

\paragraph{Application image} Alchemist è un progetto modulare ed ogni modulo è distribuito in un archivio JAR specifico. Come descritto nella documentazione\footnote{https://alchemistsimulator.github.io/howtos/preparation/jar/index.html} il software predispone due modalità di utilizzo stand-alone attraverso l'esecuzione degli archivi Java. La prima modalità consiste nell'inclusione dei singoli moduli richiesti come \textit{classpath} del processo di esecuzione. La seconda modalità sfrutta l'archivio denominato ``full", ossia un \textit{fat-jar} contenente tutti i moduli e tutte le dipendenze necessarie all'esecuzione del software in tutte le sue parti. In ottica di ridurre le dimensioni del pacchetto e velocizzare il processo di impacchettamento, jpackage costruirà l'\textit{application-image} utilizzando quest'ultimo. La posizione ed organizzazione dei file è diversa a seconda della piattaforma di destinazione del pacchetto, il risultato dell'installazione in un ambiente Linux è descritto nella figura \ref{fig:activity-interaction-diagram}.  

\begin{figure}
	\centering
	\includegraphics[width=.7\linewidth]{figures/application-image-folder-structure.pdf}
	\caption{Struttura del filesystem dell'\textit{application image} generato da \textit{jpackage} con Linux come piattaforma target}
	\label{fig:application-image-folder-structure}
\end{figure}

\paragraph{Integrazione} Per introdurre jpackage nel build system è necessario un task che funga da \textit{wrapper}: il quale esponga proprietà corrispondenti alle opzioni della \ac{cli} di jpackage. Gradle utilizza una pratica denominata \textit{lazy configuration} che fornisce la dichiarazione delle \textit{lazy properties} vale a dire ``proprietà pigre". Questa caratteristica consente di legare una proprietà ad un'altra senza doversi preoccupare dell'ordine di esecuzione. In tale modo non sono necessarie particolari attenzioni nell'assegnazione di proprietà come la versione, la quale viene calcolata da un plugin apposito. Il programma jpackage, come illustrato nella sezione \ref{sec:packaging}, non è \textit{cross-platform}, ciò significa che la generazione dei pacchetti deve essere eseguita su una macchina ospitante il sistema operativo di destinazione dei pacchetti richiesti. Per quanto lo strumento cerchi di unificare i diversi ambienti, ogni tipologia di pacchetto specialmente se di piattaforme diverse presenta limiti e regole differenti. Per questa ragione il task deve prevedere l'utilizzo di parametri differenti a seconda del sistema operativo sottostante.

\subsection{Design finale} Il design ultimo è raffigurato nella figura \ref{fig:gradle-jpackage-scheme}. 

\begin{figure}[htb]
	\centering
	\includegraphics[width=.8\linewidth]{figures/gradle-jpackage-scheme.pdf}
	\caption{Diagramma delle attività rappresentante il processo di generazione dei pacchetti}
	\label{fig:gradle-jpackage-scheme}
\end{figure}
\noindent Come discusso nella precedente sezione il design divide la generazione nei tre sistemi operativi target e la sua esecuzione dipende da \texttt{al\-che\-mi\-st\--fu\-ll\-:sha\-dow\-Jar\-}: il task incaricato di generare l'archivio JAR full. In questo modo, quando il task incaricato di generare i pacchetti con jpackage viene invocato, in primo luogo Gradle genera l'archivio JAR necessario a jpackage per costruire un application image valido.

\section{Pipeline}
La pipeline è l'elemento chiave per generare l'automazione dei processi descritti. Alchemist è già provvisto di una pipeline, la quale si occupa di analizzare il codice, verificare i processi di rilascio e in caso fosse necessario rilasciare una nuova versione. Il compito del design esplicato in questa sezione è quello di introdurre nuovi step con lo scopo di automatizzare la generazione e distribuzione dei pacchetti generati precedentemente. Le diverse unità di esecuzione o \textit{job} che popolano la pipeline possono essere distinti in base al loro ruolo.
\begin{itemize}
	\item \textbf{Inizializzazione}, tutte le unità incaricate di preparare l'ambiente di esecuzione della pipeline e quindi dei successivi job. 
	\item \textbf{Build e analisi}, i job responsabili di analizzare, compilare ed eseguire i test del codice.
	\item \textbf{Assemblaggio}, le unità di esecuzione responsabili della creazione di artefatti: archivi, pacchetti e documentazione.
	\item \textbf{Test}, job i quali verificano la validità degli artefatti o operazioni come la distribuzione.
	\item \textbf{Rilascio}, i componenti incaricati al rilascio di una nuova versione del software e le relative operazioni accessorie.
\end{itemize}
\begin{figure}[htb]
	\centering
	\includegraphics[width=.5\linewidth]{figures/pipeline-roles.pdf}
	\caption{Diagramma dell'attività illustrante il flusso attraverso i ruoli delineati}
	\label{fig:workflow-roles-summary}
\end{figure}
Il flusso è raffigurato nella figura \ref{fig:workflow-roles-summary}. L'\textbf{inizializzazione} trova spazio al primo posto e la sua esecuzione è strettamente necessaria per il proseguimento del flusso. L'\textbf{assemblaggio} e \textbf{build} sono eseguiti parallelamente mentre i \textbf{test} devono inevitabilmente dipendere dalla fase di assemblaggio per poter verificare l'output prodotto da quest'ultimo. Il \textbf{rilascio} infine, richiede che tutte le fasi descritte precedentemente siano eseguite con successo. I ruoli concernenti il lavoro descritto da questo elaborato sono: assemblaggio, test e rilascio.

\subsection{Integrazione}

\paragraph{Interazione con il Build System} Per conseguire gli obiettivi dettati dai ruoli di assemblaggio e test è necessario l'utilizzo del build system. In particolare Alchemist sfrutta lo script \textit{gradle wrapper} per interagire con esso. Il file \textit{gradlew} è uno script che permette di eseguire Gradle senza doverlo installare globalmente: alla prima esecuzione controlla la versione richiesta definita in un file di configurazione, se quest'ultima non è presente allora il wrapper scarica questa e la utilizza per eseguire i task richiesti. I job, le unità di esecuzione della piattaforma GitHub Actions, consentono l'esecuzione di comandi nella \textit{shell} default (oppure una differente) del sistema operativo presente nel runner. In questo modo tramite comandi da shell è possibile eseguire lo script e fornire i task quali vogliamo eseguire come argomenti di questo.

\paragraph{Test delle funzionalità} Lo \textit{status} di un job indica il risultato dell'esecuzione di esso e può essere: \textit{failure}, \textit{success} oppure \textit{skipped}. Lo stato di un job è considerato in errore nel caso l'esecuzione di un comando restituisca un valore diverso da 0 e viceversa di successo nel caso restituisca 0, perciò non sono necessarie particolari funzionalità per implementare un processo di verifica. Lo stato \textit{skipped}, invece, si riferisce ai job non eseguiti secondo condizioni specifiche descritte dallo sviluppatore, è quindi possibile gestire un unico workflow e modificare il suo flusso a seconda dell'evento che ha innescato l'esecuzione. Il comportamento dei job di default all'interno di una pipeline è bloccante per cui il fallimento di uno porta all'interruzione dell'intera pipeline, come raffigurato nel diagramma \ref{fig:activity-diagram-job}.
\begin{figure}[htb]
	\centering
	\includegraphics[width=.18\linewidth]{figures/activity-diagram-job.pdf}
	\caption{Diagramma dell'attività illustrante il comportamento dei job all'interno della pipeline}
	\label{fig:activity-diagram-job}
\end{figure}

\subsection{Risultato}

Lo schema risultante presenta diverse nuove attività: nel ruolo di assemblaggio la generazione dei pacchetti, nella fase di test numerose verifiche che controllano la validità di pacchetti e la pubblicazione di questi nei repository, infine nel rilascio le azioni necessarie alla pubblicazione.

%FIGURA

\section{Release}
La release di un software ricopre una fase fondamentale all'interno di un contesto \ac{cicd}. Essa integra tutte le modifiche apportate al software (previa validazione e test automatico di queste) dalla precedente release fino all'istante in cui la nuova viene creata. Successivamente al rilascio è necessario distribuire il software attraverso supporti di installazione adeguati e diffonderlo efficacemente per assicurare all'utente l'utilizzo della versione più aggiornata dell'applicativo.

\subsection{Repository}
Nell'ambito dei package management system, i repository sono i database online da cui i gestori di pacchetti reperiscono gli applicativi. Più genericamente, si intende un qualsiasi archivio online da cui è possibile scaricare software. Ogni repository supporta tipologie di pacchetti e modalità di pubblicazione differenti, le quali sono discusse nella seguente sezione.

\paragraph{AUR} L'Arch User Repository richiede la compilazione di un file PKGBUILD, ossia uno script contenente gli step necessari a scaricare, estrarre, compilare ed infine installare il software. Lo script segue uno schema predefinito con parametri obbligatori ed altri facoltativi che modificano i meta-dati del pacchetto risultante. Oltre alla presenza di diversi attributi standard come: nome, versione, descrizione, licenza ed altri, attraverso delle funzioni predefinite è possibile modificare il processo di installazione utilizzando il linguaggio bash. Una volta eseguito il PKGBUILD attraverso il comando \texttt{makepkg}, questo clona il repository o scarica il pacchetto indicato come sorgente, successivamente sono le funzioni a determinare come dall'input ottenuto si ricava il programma installato nel sistema.
Le funzioni sono eseguite nel seguente ordine: (i) prepare, eseguita appena dopo che è stato scaricato ed estratto il pacchetto indicato nella sorgente, (ii) pkgver, utilizzata per stabilire la versione del pacchetto se questa non è già stata impostata direttamente, (iii) build, i comandi necessari a compilare il software (se necessario), (iv) check, verifica che gli step precedenti siano stati eseguiti correttamente ed infine (v) package, l'installazione dei file nel filesystem. Per non modificare direttamente il filesystem del computer installante, \texttt{makepkg} crea due directory \textit{pkg} e \textit{src}. All'interno di src, esso estrae il pacchetto e tutto il suo contenuto, mentre pkg consiste in un ambiente che simula il filesystem del sistema. In fase di installazione sarà il package manager a replicare la struttura della directory pkg nel filesystem del sistema installante.

Una volta scritto lo script il pacchetto è pronto per essere installato e quindi pubblicato. Attraverso un programma fornito da Arch di nome \texttt{namcap} è possibile controllare che la sintassi ed i valori inseriti nello script siano validi. I pacchetti all'interno dell'\ac{aur} consistono in repository git contenenti il PKGBUILD ed altri file di configurazione opzionali, il processo di pubblicazione dunque è simile a qualsiasi progetto con un sistema di version control: la creazione di un commit e la pubblicazione di esso.

\paragraph{Winget} Il package manager winget presenta una struttura simile, un pacchetto è formato da diversi file \textit{manifest} i quali descrivono i meta-dati del pacchetto nel linguaggio YAML. A differenza del PKGBUILD, non ci sono script e non esistono funzioni, l'intera configurazione è descrittiva e non presenta la possibilità di inserire comandi da eseguire pre o post installazione. I file manifest si distinguono in: manifesto della versione, contenente dettagli identificativi del pacchetto, il manifesto delle impostazioni locali, il quale descrive la configurazione per uno specifico locale, ed il manifesto dell'installer, contenente l'URL dove reperire il pacchetto installante ed altre informazioni su di esso. Per semplificare il processo di creazione e pubblicazione dei pacchetti, Microsoft prevede l'utilizzo di uno script ``wingetcreate" che guida l'utente nella scelta dei parametr. Inoltre si presta ad essere utilizzato all'interno di pipeline \ac{cicd} per aggiornare pacchetti già presenti.

\subsection{Semantic release}
Il processo di \ac{cicd} utilizzato da Alchemist, prevede l'utilizzo di una tecnica chiamata \textit{semantic release}, legata con il più conosciuto concetto di \textit{semantic versioning}. Abbreviato come ``SemVer", esso è uno schema di versionamento standardizzato per determinare le versioni di un software. È stato progettato per rendere intuitivo comprendere le modifiche apportate al software ed il loro impatto riguardo la compatibilità con le versioni precedenti. Lo schema descrive una versione come tre cifre principali separate da punti.
\begin{itemize}
	\item La prima cifra, \textbf{major}, si incrementa quando vengono apportate modifiche sostanziali al software che lo rendono incompatibile con le versioni precedenti.
	\item La seconda, \textbf{minor}, indica l'aggiunta di nuove funzionalità o miglioramenti al software senza eliminare la retro-compatibilità di questo.
	\item La terza cifra, \textbf{patch}, viene incrementata quando una nuova versione risolve bug o problemi di sicurezza.
\end{itemize}
Oltre allo schema principale, SemVer prevede anche la presenza di etichette che stabiliscono se la versione è in fase di sviluppo(alpha, beta) oppure di test.
Attraverso l'utilizzo dei \textit{conventional commits}, il progetto è capace di: stabilire la versione autonomamente, elencare le modifiche tra un rilascio e quello successivo in un \textit{changelog} e decidere quando è necessaria la pubblicazione di una nuova versione. I conventional commits sono un insieme di regole che riguardano i messaggi dei commit, ossia le parti di testo allegate alla modifica del software che lo sviluppatore annette. Le regole stabiliscono una sintassi predefinita che permette di comprendere attraverso la cronologia dei commit le modifiche apportate al software, ed inoltre assieme allo schema SemVer permette a tool automatici di calcolare la versione.

\paragraph{Limitazioni} La creazione dei rilasci avviene attraverso un plugin che opera all'interno di un job dedicato nella pipeline. Il job ``Release": 
\begin{enumerate}
	\item invoca il processo di rilascio eseguendo \texttt{npx semantic-release},
	\item il plugin controlla se è necessario eseguire un rilascio e in caso positivo procede caricando la configurazione descritta nel file ``release.config.js",
	\item dopo aver creato la release su GitHub con gli assets evidenziati, esegue il comando descritto nella configurazione.
\end{enumerate}
L'unica sezione espandibile nella quale è possibile determinare le azioni svolte durante una pubblicazione è il comando configurato. Questo dettaglio crea delle restrizioni: (i) sul sistema operativo, perché solo un job può prendere in carico il lavoro di rilascio e di conseguenza un solo runner con uno specifico sistema operativo esegue il comando, (ii) la pubblicazione dei pacchetti deve avvenire mediante l'utilizzo di comandi da shell.

\paragraph{Soluzione} Il job incaricato al rilascio utilizza Linux, mentre la generazione dei manifest per winget deve essere eseguita in una macchina ospitante Windows. Mediante l'esecuzione del comando di pubblicazione descritto nella configurazione,  la shell imposta una variabile d'ambiente la quale comunica se il rilascio è stato eseguito correttamente. Quest'ultima viene emessa in output in modo che sia leggibile all'interno della pipeline. A seconda dell'output del job ``Release", un secondo job adibito al rilascio su winget viene eseguito o saltato. In questo modo non modifichiamo l'attuale architettura del flusso, lasciando intatto il rilascio gestito attraverso il plugin, ma estendendolo mediante le funzionalità che GitHub Actions fornisce. La procedura di aggiornamento per \ac{aur}, invece, è riassumibile in uno script bash e eseguibile quindi attraverso il comando configurato nel plugin.